%%%%%%%%%%%%%%%%%%%%%%%%%%%%%%%%%%%%%%%%%%%%%%%%%%%%%%%%%%%%%%%%%%%%%%%%%%%%
% AGUJournalTemplate.tex: this template file is for articles formatted with LaTeX
%
% This file includes commands and instructions
% given in the order necessary to produce a final output that will
% satisfy AGU requirements, including customized APA reference formatting.
%
% You may copy this file and give it your
% article name, and enter your text.
%
%
% Step 1: Set the \documentclass
%
% There are two options for article format:
%
% PLEASE USE THE DRAFT OPTION TO SUBMIT YOUR PAPERS.
% The draft option produces double spaced output.
%

%% To submit your paper:
\documentclass[draft]{agujournal2019}
\usepackage{url} %this package should fix any errors with URLs in refs.
\usepackage{lineno}
\usepackage{color}
\graphicspath{ {figures/} }
\linenumbers
%%%%%%%
% As of 2018 we recommend use of the TrackChanges package to mark revisions.
% The trackchanges package adds five new LaTeX commands:
%
%  \note[editor]{The note}
%  \annote[editor]{Text to annotate}{The note}
%  \add[editor]{Text to add}
%  \remove[editor]{Text to remove}
%  \change[editor]{Text to remove}{Text to add}
%
% complete documentation is here: http://trackchanges.sourceforge.net/
%%%%%%%

\draftfalse

\journalname{JGR: Space Physics}


\begin{document}

\title{Statistical Properties of Curtain Electron Precipitation Derived with AeroCube-6}

%% ------------------------------------------------------------------------ %%
%
%  AUTHORS AND AFFILIATIONS
%
%% ------------------------------------------------------------------------ %%

\authors{M. Shumko\affil{1}, A.T. Johnson\affil{1}, J.G. Sample\affil{1}, D.L. Turner\affil{3}, T.P. O'Brien\affil{2}, and J.B. Blake\affil{2}}


\affiliation{1}{Department of Physics, Montana State University, Bozeman, Montana, USA}
\affiliation{2}{Space Science Applications Laboratory, The Aerospace Corportation, El Segundo, California USA}
\affiliation{3}{Johns Hopkins Applied Physics Laboratory, Laurel, Maryland, USA}

\correspondingauthor{M. Shumko}{msshumko@gmail.com}

\begin{keypoints}
\item We used the dual AeroCube-6 CubeSats to identify stationary, narrow, and persistent $>30$ keV precipitation in low Earth orbit
\item A single low Earth-orbiting spacecraft can easily misidentify curtains as microburst precipitation
\item A few curtains were persistently scattered into the atmosphere for at least six seconds
\end{keypoints}

%% ------------------------------------------------------------------------ %%
%
%  ABSTRACT
%
% A good abstract will begin with a short description of the problem
% being addressed, briefly describe the new data or analyses, then
% briefly states the main conclusion(s) and how they are supported and
% uncertainties.
%% ------------------------------------------------------------------------ %%

%% \begin{abstract} starts the second page

\begin{abstract}
\end{abstract}

\section{Plain Language Summary}

\section{Introduction}
\textcolor{blue}{
Outline
\begin{enumerate}
\item Introduce various particle loss mechanisms
\item Introduce microbursts and their effect on atmospheric chemistry. Maybe mention how there is an unexplained source of HOX and NOX?
\item Introduce curtains and the prevailing hypothesis linking curtains to microbursts
\item If curtains are drifting then we have overestimated the atmospheric losses due to microbursts 
\item Our goal is to study three statistical properties of curtains: location, spatial width, and preferred geomagnetic conditions. Lastly we will use the SAA to determine if some curtains were drifting around the Earth or locally and persistently precipitating
\item Explain DLC and BLC. DLC description from Comes 2003 paper and maybe something from Craig Rogers group.
\item Maybe cite the curtain paper from 2000 that relates them to lightning? Title: Trapped energetic electron curtains produced by thunderstorm driven relativistic runaway electrons
\end{enumerate}
}

\section{Instrumentation} \label{instrumentation}
\textcolor{blue}{Think about the flow, and avoid plagiarizing myself}

The AC6 mission was a pair of 0.5U (10x10x5 cm) CubeSats built by The Aerospace Corporation designed to measure the electron and proton environment in low Earth orbit \cite{O'brien2016}. AC6 was launched on 19 June 2014 into a 620x700 km, $98^\circ$ inclination orbit. The AC6 orbit over the three year mission lifetime was roughly dawn-dusk, and precessed only a few hours in MLT; 8-12 MLT in dawn and 20-24 MLT in dusk. The two AC6 spacecraft, designated as AC6-A and AC6-B, separated after launch and were in proximity for the duration of the three year mission---maintained by an active attitude control system. The attitude control system allowed then to precisely control the amount of atmospheric drag experienced by each AC6 unit using the surface area of their solar panel ``wings". By changing their orientation, AC6 was able to maintain a separation between 2-800 km, confirmed with the Global Positioning System. The two AC6 units were in a string of pearls configuration so one unit, typically unit A, was leading the other by an in-track lag---the time it would take the following spacecraft to catch up to the position of the leading spacecraft. To convert between the AC6 in-track separation and in-track lag, we assume a typical 7.5 km/s orbital velocity of LEO spacecraft. The in-track lag was readily available with the Global Positioning System which makes it easy to study precipitation phenomena observed at the same time, and at the same position by shifting one time series by the in-track lag.

Each AC6 unit contains three Aerospace microdosimeters (licensed to Teledyne Microelectronics, Inc) that measure the electron and proton dose in orbit \cite{O'brien2016}. The dosimeter used for this study is dos1 with a $30$ keV electron threshold. dos1 is used for this study because the other dosimeters were not identical between unit A and B. All dosimeters sample at 1 Hz in survey mode, and 10 Hz in burst mode. 10 Hz data was readily available from both AC6 units from June 2014 to May 2017 while their in-track lag was less than 65 seconds, and at times was a fraction of a second. \textcolor{red}{Show a distribution of the in-track lag when they had 10 Hz data?} The variety of AC6 separations and data availability over the three-year mission makes it possible to study tranisent electron microburst precipitation \cite{Shumko2019} and now stationary electron curtain precipitation.

\iffalse
The AC6 mission consists of a pair of 0.5U (10x10x5 cm) CubeSats built by The Aerospace Corporation and launched on June 19th, 2014 into a 620 x 700 km, $98^\circ$ inclination orbit. The two satellites, designated as AC6-A and AC6-B, separated after launch and drifted apart. Both AC6 units have an active attitude control system which allows them to adjust the atmospheric drag experienced by each AC6 unit by orienting their solar panel ``wings" with respect to the ram direction. By changing their orientation, the AC6 mission was able to achieve fine separation control and maintain a separation between 2-800 km, which was confirmed with GPS. Figure \ref{fig1}a shows the AC6 separation for the duration of the mission. Figure \ref{fig1}b shows where both AC6 units were taking 10 Hz data simultaneously as a function of L and MLT which highlights that most data were taken at 8-12 MLT, an ideal local time for observing microbursts. Lastly Fig. \ref{fig1}b shows that over a three year period the AC6 orbit was roughly dawn-dusk, sun-synchronous, and precessed 8 to 12 MLT in the dawn region. AC6's 8-12 MLT precession is ideal for sampling the region where microbursts are most likely to be observed \cite<e.g.>{O'Brien2003} but the tradeoff is limited microburst size information in MLT.

Each AC6 unit is equipped with three Aerospace microdosimeters (licensed to Teledyne Microelectronics, Inc). The dosimeter used for this study, dos1, is identical on both AC6 units and has a $35$ keV electron threshold. All AC6 dosimeters sample at 1 Hz in survey mode, and 10 Hz in burst mode in the radiation belts  \cite{O'brien2016}. Since microburst duration is less than a second, only the 10 Hz data was used to identify microbursts. \fi

\section{Methodology} 
\subsection{Curtain Identification} \label{curtain_identification}
\textcolor{blue}{
Outline
\begin{enumerate}
\item Shifted the AC6B time series by the in-track lag and looked for times when the following two conditions were met:
\item a one second running correlation was greater than 0.8 and
\item the correlated counts were bursty - defined as greater than two standard deviations (assuming Poission statistics) above a ten second-long running mean.
\item Events were automatically identified and checked by an author to catalog 1634 curtains. Examples shown in Fig. 1.
\item Detection method similar to Greeley 2019 and Blum 2015.
\item Various parameters were explored and we tuned it to have as many candidate events as possible while being feasible to inspect every detection. 
\item Baseline sensitivity decreases with larger structures, depending on the curtain amplitude, background level, and baseline width. Sensitivity begins to rapidly diminish for widths close to half of the baseline width---around five seconds, correspondent to 38 km size, for this identification criteria.
\end{enumerate}
}

\begin{figure}
\includegraphics[width=\textwidth]{fig1.pdf}
\caption{Four examples showing the AC6 $> 30$ keV electron data taken by AC6 at the same time in the top row and at the same position in the bottom row. AC6-A, whose data is shown with red curves, was $s$ kilometers ahead of AC6-B. To show the data at the same position the time series data from one spacecraft was shifted by the in-track lag and annotated by dt. These examples show curtain precipitation that was highly correlated for up to 26 seconds.}
\label{fig1}
\end{figure}

\section{Results} \label{results}
\textcolor{blue}{
Outline
\begin{enumerate}
\item Show curtain width and comment how narrow they are. Whether they are drifting or locally precipitating, they must have a very filamentary structure that persists for multiple seconds
\item Figure out how the detection bias affects the width distribution
\item Show, and comment on the Auroral electroject strengths when each curtain was observed. Curtains are more likeliy to be observed during disturbed times.
\item Discuss the SAA, BLC, and show the curtains the the BLC plot. Mention how these electrons must have been precipitating for multiple seconds, over an order of magnitude longer than typical microbursts.
\end{enumerate}
}

In the spirit of brevity, we limited the scope of these results to answer the following three questions:

\begin{enumerate}
\item how narrow are curtains,
\item when and where are curtains observed, and
\item are curtains drifting or locally precipitating?
\end{enumerate}

\begin{figure}
\includegraphics[width=\textwidth]{ac6_curtain_microburst_width_dist.pdf}
\caption{Size distributions of curtains (AC6 in-track separation mostly in latitude) in black and microbursts in red as a function of AC6 in-track width. Microburst distribution adopted from \citeA{Shumko2019}.}
\label{ae_width_dist}
\end{figure}

\begin{figure}
\includegraphics[width=\textwidth]{ac6_curtain_AE_dist.pdf}
\caption{The distribution of the Auroral Electroject index from 2014 to 2017 shown by the thick black curve, and the Auroral Electroject index when curtains were observed by the red curve.}
\label{ae_width_dist}
\end{figure}

\begin{figure}
\includegraphics[width=\textwidth]{fig2.pdf}
\caption{Distribution of curtains as a function of L and MLT. To avoid noisy normalization scaling, bins with less than $10,000$ 10 Hz samples were not normalized in panel b.}
\end{figure}

\begin{figure}
\includegraphics[width=\textwidth]{fig2_2.pdf}
\caption{Distribution of curtains as a function of L and MLT. White bins in panels a and c have 0 curtain detections or 10 Hz samples. To avoid noisy normalization scaling, bins with less than $10,000$ 10 Hz samples were not normalized in panel b. \textcolor{red}{Or show this version?}}
\end{figure}

\begin{figure}
\includegraphics[width=\textwidth]{curtain_L_vs_AE.pdf}
\caption{\textcolor{red}{What about this figure?}}
\end{figure}

\subsection{Local Atmospheric Precipitation}
\begin{figure}
\includegraphics[width=\textwidth]{fig3.pdf}
\caption{Curtains observed inside the bounce loss cone.}
\label{fig3}
\end{figure}

\section{Discussion} \label{discussion}
\textcolor{blue}{
Outline
\begin{enumerate}
\item Curtains are spatially small and must be around a few hundred km at the equator
\item curtain phenomena originates in the outer radiation belt, and observed relatively more in the evening than morning regions. Limited AC6 coverage prevents a complete MLT distribution
\item preference to disturbed conditions
\item some curtains locally precipitate for an extended period of time so there must be a sustained parallel electric field. Show the derivation and estimated potential.
\item AC6 can't answer this question, but curtains could provide a substantial source of HOx and NOX molecules responsible for destroying ozone. We need AC6 with energy and pitch angle resolution.
\end{enumerate}}

\section{Conclusions}

% \appendix

\acknowledgments
This work was made possible with the help from the many engineers and scientists at The Aerospace Corporation who designed, built, and operated AC6. M. Shumko was supported by NASA Headquarters under the NASA Earth and Space Science Fellowship Program - Grant 80NSSC18K1204. D.L. Turner is thankful for support from the Van Allen Probes mission and a NASA grant (Prime award number: 80NSSC19K0280). The work at The Aerospace Corporation was supported in part by RBSP-ECT funding provided by JHU/APL contract 967399 under NASA's Prime contract NAS501072. The AC6 data is available at http://rbspgway.jhuapl.edu/ac6 and the IRBEM-Lib version used for this analysis can be downloaded from https://sourceforge.net/p/irbem/code/616/tree/.

\section{Homeless Words}

Title: Statistical Properties of Curtains--Latitudinally-Narrow and Persistent Electron Precipitation Phenomena

This study leverages AC6, a multi-spacecraft mission, to interpret and understand particle precipitation in a way that is impossible with a single spacecraft.

This study leverages the asymmetry in Earth's magnetic field. The asymmetric magnetic field results in the SAA and the BLC, two very related and unique regions

Particles that impact the atmosphere are lost during that bounce motion. We found curtains in the bounce loss cone, a region in the North Atlantic near and above Iceland.

The bounce loss cone is magnetically connected to the SAA, where Earth's magnetic field is weakest near Earth's surface. A particle observed in the blc in the northern hemisphere will descend below 100 km altitude. At sub-100 km altitudes the particle has a high chance of encountering and scattering with the atmosphere and be lost. 

We found curtain electrons that, when given the chance to execute their cyclical bounce motion, will descend below Earth's surface in the SAA. An electrons can not survive that trip.

Write the paper and ask the question: "What is this paper really about?" Not just curtains, but uncovering something unexpected that has been observed and overlooked for decades.

Are curtains related to aurora? This is a good question---one that is not pertinent here (idea from The Elements of Style p.68).

Here are two parting questions that are not considered here. Why were some curtains shifted slightly? Perhaps it was due to the movement of the magnetic field lines. Also do curtains have a corresponding visual signature on the ground? The answer to this question will show if curtains are related to the aurora.

\bibliography{/home/mike/Dropbox/0_firebird_research/A_presentations/refs}
%\bibliography{"refs"}

\end{document}